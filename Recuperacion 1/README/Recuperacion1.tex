\documentclass{article}
\usepackage{algorithm2e}
\usepackage{color}
\usepackage{listings}
\usepackage{setspace}

\definecolor{Code}{rgb}{0,0,0}
\definecolor{Decorators}{rgb}{0.5,0.5,0.5}
\definecolor{Numbers}{rgb}{0.5,0,0}
\definecolor{MatchingBrackets}{rgb}{0.25,0.5,0.5}
\definecolor{Keywords}{rgb}{0,0,1}
\definecolor{self}{rgb}{0,0,0}
\definecolor{Strings}{rgb}{0,0.63,0}
\definecolor{Comments}{rgb}{0,0.63,1}
\definecolor{Backquotes}{rgb}{0,0,0}
\definecolor{Classname}{rgb}{0,0,0}
\definecolor{FunctionName}{rgb}{0,0,0}
\definecolor{Operators}{rgb}{0,0,0}
\definecolor{Background}{rgb}{0.98,0.98,0.98}
\lstdefinelanguage{Python}{
	numbers=left,
	numberstyle=\footnotesize,
	numbersep=1em,
	xleftmargin=1em,
	framextopmargin=2em,
	framexbottommargin=2em,
	showspaces=false,
	showtabs=false,
	showstringspaces=false,
	frame=l,
	tabsize=4,
	% Basic
	basicstyle=\ttfamily\small\setstretch{1},
	backgroundcolor=\color{Background},
	% Comments
	commentstyle=\color{Comments}\slshape,
	% Strings
	stringstyle=\color{Strings},
	morecomment=[s][\color{Strings}]{"""}{"""},
	morecomment=[s][\color{Strings}]{'''}{'''},
	% keywords
	morekeywords={import,from,class,def,for,while,if,is,in,elif,else,not,and,or,print,break,continue,return,True,False,None,access,as,,del,except,exec,finally,global,import,lambda,pass,print,raise,try,assert},
	keywordstyle={\color{Keywords}\bfseries},
	% additional keywords
	morekeywords={[2]@invariant,pylab,numpy,np,scipy},
	keywordstyle={[2]\color{Decorators}\slshape},
	emph={self},
	emphstyle={\color{self}\slshape},
	%
}
\linespread{1.3}

\newenvironment{answer}[0]{}

\newenvironment{changemargin}[2]{%
	\begin{list}{}{%
			\setlength{\topsep}{0pt}%
			\setlength{\leftmargin}{#1}%
			\setlength{\rightmargin}{#2}%
			\setlength{\listparindent}{\parindent}%
			\setlength{\itemindent}{\parindent}%
			\setlength{\parsep}{\parskip}%
		}%
		\item[]}{\end{list}}
	
\title{Mergesort usando computación distribuida}
\author{Victor Farfan}

	
\begin{document}
	\maketitle	
	\newpage
	\section{Readme\textsl{}}
		{\it mergesort{\_}input.txt} es el archivo donde van las oraciones que queremos ordenar.\\
		{\it save.txt} es el archivo donde se van a guardar las oraciones ya ordenadas.\\
		
		\begin{itemize}
			\item Descargar los archivos que se encuentran en el repositorio en la carpeta Recuperacion1 y entrar a la carpeta Code
			\item Hacer build de la imagen de docker con el comando \it docker build -t \textless tag propio \textgreater .
			\item Levantar el contenedor recien creado con el comando docker run -it \$(pwd):/code \textless tag propio \textgreater  jug execute jugfile.py
		\end{itemize}
		Podemos estar seguros del resultado al ver que se imprime en consola la lista de oraciones ordenada. Esta impresion se hace leyendo el archivo despues de haberle escrito encima e imprimiendo cada linea de su contenido. 
		Nuestra otra opcion para comprobar que se escribio la data correcta en el archivo es levantar el contenedor con el comando {\it docker run -it \$(pwd):/code \textless tag propio \textgreater  sh}. Con esto entramos a la terminal dentro del contenedor. Estando en la carpeta "code" ejecutamos {\it cat save.txt} para verificar el contenido del archivo en donde se guardaron las oraciones ya ordenadas. 
	\begin{answer}
		
	\end{answer}
	

	
	
	

\end{document}