\documentclass{article}
\usepackage{algorithm2e}
\begin{document}

\title{Problema \#2}
\author{Victor Farfan}
\maketitle
	\begin{algorithm}[H]
	\KwIn{matriz A (n x m) y matriz B (m x p)}
	\KwOut {matriz C (n x p)}
	\For {i from 1 to n}{
		\For{j from 1 to p:}{
			sum = 0\\
			\For{k from 1 to m:}{
				sum = sum + A[i][k] * B[k][j]
			}
			C[i][j] = sum
		}
	}
	\KwRet {C}\newline
	\end{algorithm}
	
\newenvironment{answer}
\begin{answer}
	
El running time de este algoritmo depende de los tamaños n,m y p de las matrices. Por cada recorrido en el ciclo de "n" se hace un ciclo entero de 1 a p. Por cada recorrido en el ciclo de "p" se hace un ciclo enter de 1 a m. Por lo que el tiempo de ejecución sería O(n*m*p). Si las matrices tienen las mismas dimensiones el tiempo de ejecucición es \begin{math} O(n^{3}) \end{math}

\end{document}