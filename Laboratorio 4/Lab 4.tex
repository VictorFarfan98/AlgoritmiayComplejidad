\documentclass{article}
\usepackage{algorithm2e}
\usepackage{color}
\usepackage{listings}
\usepackage{setspace}
\usepackage{graphicx}

\definecolor{Code}{rgb}{0,0,0}
\definecolor{Decorators}{rgb}{0.5,0.5,0.5}
\definecolor{Numbers}{rgb}{0.5,0,0}
\definecolor{MatchingBrackets}{rgb}{0.25,0.5,0.5}
\definecolor{Keywords}{rgb}{0,0,1}
\definecolor{self}{rgb}{0,0,0}
\definecolor{Strings}{rgb}{0,0.63,0}
\definecolor{Comments}{rgb}{0,0.63,1}
\definecolor{Backquotes}{rgb}{0,0,0}
\definecolor{Classname}{rgb}{0,0,0}
\definecolor{FunctionName}{rgb}{0,0,0}
\definecolor{Operators}{rgb}{0,0,0}
\definecolor{Background}{rgb}{0.98,0.98,0.98}
\lstdefinelanguage{Python}{
	numbers=left,
	numberstyle=\footnotesize,
	numbersep=1em,
	xleftmargin=1em,
	framextopmargin=2em,
	framexbottommargin=2em,
	showspaces=false,
	showtabs=false,
	showstringspaces=false,
	frame=l,
	tabsize=4,
	% Basic
	basicstyle=\ttfamily\small\setstretch{1},
	backgroundcolor=\color{Background},
	% Comments
	commentstyle=\color{Comments}\slshape,
	% Strings
	stringstyle=\color{Strings},
	morecomment=[s][\color{Strings}]{"""}{"""},
	morecomment=[s][\color{Strings}]{'''}{'''},
	% keywords
	morekeywords={import,from,class,def,for,while,if,is,in,elif,else,not,and,or,print,break,continue,return,True,False,None,access,as,,del,except,exec,finally,global,import,lambda,pass,print,raise,try,assert},
	keywordstyle={\color{Keywords}\bfseries},
	% additional keywords
	morekeywords={[2]@invariant,pylab,numpy,np,scipy},
	keywordstyle={[2]\color{Decorators}\slshape},
	emph={self},
	emphstyle={\color{self}\slshape},
	%
}
\linespread{1.3}

\newenvironment{answer}[0]{}

\newenvironment{changemargin}[2]{%
	\begin{list}{}{%
			\setlength{\topsep}{0pt}%
			\setlength{\leftmargin}{#1}%
			\setlength{\rightmargin}{#2}%
			\setlength{\listparindent}{\parindent}%
			\setlength{\itemindent}{\parindent}%
			\setlength{\parsep}{\parskip}%
		}%
		\item[]}{\end{list}}
	
\title{Laboratorio \#3}
\author{Victor Farfan}

	
\begin{document}
	\maketitle	
	\section{Problema \#1 - Método de Sustitución}
	
	\begin{answer}
		\subsection{Inciso 1: \begin{math} T(n-1) + n = O(n^2)\end{math}}
			\begin{math} T(n) <= c(n-1)^2 + n \end{math}\\
			\begin{math} T(n) <= c(n^2 - 2n + 1) + n \end{math}\\
			\begin{math} T(n) <= cn^2 - 2cn + c + n \end{math}\\
			\begin{math} T(n) <= cn^2 \end{math}\\
			\begin{math} T(n) <= n^2 \end{math}
			
		\subsection{Inciso 2: \begin{math} T(n/2) + 1 = O(log(n))\end{math}}
			\begin{math} T(n) <= c(log(n/2)) + 1 \end{math}\\
			\begin{math} T(n) <= c*log(n) - c*log(2) + 1 \end{math}\\
			\begin{math} T(n) <= c*log(n) \end{math}\\
			\begin{math} T(n) <= log(n) \end{math}\\
	\end{answer}
	
	\newpage
	\section{Problema \#2 - Método de Árbol Recursivo}
	\begin{answer}
		\subsection{Inciso 1: \begin{math} T(n) = 3T(n/2) + n\end{math}}
		
		\includegraphics[scale=0.3]{tree}\\
		
		\begin{math} T(n) <= 3(c(log(n/2))) + n \end{math}\\
		\begin{math} T(n) <= 3cn +n \end{math}\\
		\begin{math} T(n) <= n + n \end{math}\\
		\begin{math} T(n) <= n \end{math}
	\end{answer}
	
	\newpage
	\section{Problema \#3}
	
	

\end{document}